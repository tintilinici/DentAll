\chapter{Arhitektura i dizajn sustava}
		
		\textbf{\textit{dio 1. revizije}}\\

		\textit{ Potrebno je opisati stil arhitekture te identificirati: podsustave, preslikavanje na radnu platformu, spremišta podataka, mrežne protokole, globalni upravljački tok i sklopovsko-programske zahtjeve. Po točkama razraditi i popratiti odgovarajućim skicama:}
	\begin{itemize}
		\item 	\textit{izbor arhitekture temeljem principa oblikovanja pokazanih na predavanjima (objasniti zašto ste baš odabrali takvu arhitekturu)}
		\item 	\textit{organizaciju sustava s najviše razine apstrakcije (npr. klijent-poslužitelj, baza podataka, datotečni sustav, grafičko sučelje)}
		\item 	\textit{organizaciju aplikacije (npr. slojevi frontend i backend, MVC arhitektura) }		
	\end{itemize}

	
		

		

				
		\section{Baza podataka}

			Za potrebe naše aplikaicje korisiti ćemo relacijsku bazu podataka kako bismo lakše oblikovali stvarni svijet. Baza nam je potrbna za metodičku pohranu podataka te njihovo brzo dohvaćanje. Naša baza podataka se sastoji od sljedećih entiteta:
			
			\begin{packed_item}
				\item Patient
				\item Accommodation
				\item AccommodationOrder
				\item AccommodationBooking
				\item TransportCompany
				\item TransportVehicle
				\item TransportBooking
				\item MedicalAppointment
				\item AdminRole
				\item AdminRoles
				\item Admin
			\end{packed_item}
		
			\subsection{Opis tablica}
			
				
				\noindent
				\textbf{Patient} Ova entitet sadrži sve informacije o korisniku usluga našeg sustava. Kako su svi korisnici pacijenti odlučili smo ih imenovati tako. Entitet sadrži atribute: patient\_ID, first\_name, last\_name, PIN, email i phone\_number. Ovaj entitet je u vezi \textit{One-to-Many} s entitetom AccommodationOrder preko atributa patient\_ID i u \textit{One-to-Many} s entitetom AccommodationBooking preko atributa patient\_ID.
				
				\begin{longtblr}[
					label=none,
					entry=none
					]{
						width = \textwidth,
						colspec={|X[6,l]|X[6, l]|X[20, l]|}, 
						rowhead = 1,
					} %definicija širine tablice, širine stupaca, poravnanje i broja redaka naslova tablice
				
					\hline 
					\SetCell[c=3]{c}{\textbf{Patient}}\\
					\hline[3pt]
					\SetCell{LightGreen}patient\_ID & VARCHAR & primarni ključ tablice \\
					\hline
					first\_name	& VARCHAR &  ime pacijenta\\ 
					\hline 
					last\_name & VARCHAR &  prezime pacijenta \\ 
					\hline 
					PIN & VARCHAR	&  Personal Identification Number, kao OIB u hrvatskoj\\ 
					\hline 
					email & VARCHAR & pacijentov email\\ 
					\hline	
					phone\_number & VARCHAR & pacijentov telefonski broj \\
					\hline
				\end{longtblr}
			
				\noindent
				\textbf{Accommodation} Ovaj entitet sadržava potrebne podatke o nekom smještaju koji je smještajni administrator unio. Entitet sadrži atribute: accommodation\_ID, type, category, address, availibility\_start, availability\_end i location. Ovaj entitet je u \textit{One-to-Many} vezi s entitetom AccommodationBooking preko atributa accommodation\_ID.
				
				\begin{longtblr}[
					label=none,
					entry=none
					]{
						width = \textwidth,
						colspec={|X[10,l]|X[6, l]|X[18, l]|}, 
						rowhead = 1,
					} %definicija širine tablice, širine stupaca, poravnanje i broja redaka naslova tablice
					\hline 
					\SetCell[c=3]{c}{\textbf{Accommodation}}\\ 
					\hline[3pt]
					\SetCell{LightGreen}accommodation\_ID & VARCHAR & primarni ključ tablice\\ 
					\hline
					type & VARCHAR & versta smještaja \\
					\hline 
					category & VARCHAR & kategorija smještaja \\
					\hline
					address & VARCHAR & adresa smještaja \\
					\hline
					availability\_start & DATETIME & datum i vrijeme od kada je smještaj dostupan \\
					\hline
					availability\_end & DATETIME & datum i vrijeme do kada je smještaj dosupan \\
					\hline
					location & POINT & kordinate smještaja \\
					\hline
				\end{longtblr}
			
			\noindent
			\textbf{AccommodationOrder} Ovaj entitet se koristi za pohravnjivanje pacijentovih zahtjeva o traženom smještaju. U slučaju da traženi smještaj nije odmah dostupan zahtjev se sprema kako bih se kasnije mogao opet pogledati. Entitet sadrži atribute: accommodation\_order\_ID, arrival\_datetime, departure\_datetime, accommodation\_type, accommodation\_category i patient\_ID. Ovaj entitet je u \textit{Many-to-One} vezi s entitetom Patient preko patient\_ID atributa.
			\begin{longtblr}[
				label=none,
				entry=none
				]{
					width = \textwidth,
					colspec={|X[13,l]|X[6, l]|X[20, l]|}, 
					rowhead = 1,
				} %definicija širine tablice, širine stupaca, poravnanje i broja redaka naslova tablice
				\hline 
				\SetCell[c=3]{c}{\textbf{AccommodationOrder}}\\ 
				\hline[3pt]
				\SetCell{LightGreen}accommodation\_order\_ID & VARCHAR & primarni ključ tablice\\ 
				\hline
				arrival\_datetime & DATETIME & vrijeme dolaska pacijenta u državu, od tada mu treba smještaj\\
				\hline
				departure\_datetime & DATETIME & vrijeme odlaska pacijenta iz države, do tada treba smještaj \\
				\hline
				accommodation\_type & VARCHAR & željeni tip smještaja koji pacijent traži \\
				\hline
				accommodation\_category & VARCHAR & željena kategorija smještaja koju pacijent traži \\
				\hline
				\SetCell{LightBlue} patient\_ID	& STRING & ID pacijenta koji je napravio ovaj zahtjev \\
				\hline 
			\end{longtblr}
			
			\noindent
			\textbf{AccommodationBooking} Ovaj vezni entitet se koristi za pohranjivanje informacije koji pacijent je kada u kojem smještaju. Entitet sadrži atribute: accommodation\_booking\_ID, start\_datetime, end\_datetime, accommodation\_ID, patient\_ID. Ovaj entitet je u \textit{Many-to-One} vezi s entitetom Accommodation preko atributa accommodation\_ID. U \textit{Many-to-One} vezi s entitetom Patient preko atributa patient\_ID. I u \textit{One-to-Many} s entitetom TransportBooking preko accommodation\_booking\_ID.
			\begin{longtblr}[
				label=none,
				entry=none
				]{
					width = \textwidth,
					colspec={|X[10, l]|X[6, l]|X[20, l]|}, 
					rowhead = 1,
				} %definicija širine tablice, širine stupaca, poravnanje i broja redaka naslova tablice
				\hline 
				\SetCell[c=3]{c}{\textbf{AccommodationBooking}}\\ 
				\hline[3pt]
				\SetCell{LightGreen}accommodation\_
				booking\_ID & VARCHAR & primarni ključ tablice \\ 
				\hline
				start\_datetime & DATETIME & datum i vrijeme od kada je pacijent u smještaju \\
				\hline
				end\_datetime & DATETIME & datum i vrijeme do kada je pacijent u smještaju \\
				\hline 
				\SetCell{LightBlue} accommodation\_ID	& VARCHAR & smještaj u koji je pacijent smješten \\
				\hline 
				\SetCell{LightBlue} patient\_ID & VARCHAR & pacijent koji ostaje u smještaju \\
				\hline
			\end{longtblr}
			
			\noindent
			\textbf{TransportCompany} Ovaj entitet sadržava informacije o transportnoj firmi. Entitet sadrži atribute: transport\_company\_ID, name, phone\_number, email. Ovaj entitet je u \textit{One-to-Many} vezi s entitetom TransportVehicle preko atributa transport\_company\_ID.
			\begin{longtblr}[
				label=none,
				entry=none
				]{
					width = \textwidth,
					colspec={|X[11,l]|X[6, l]|X[20, l]|}, 
					rowhead = 1,
				} %definicija širine tablice, širine stupaca, poravnanje i broja redaka naslova tablice
				\hline 
				\SetCell[c=3]{c}{\textbf{TransportCompany}}\\ 
				\hline[3pt]
				\SetCell{LightGreen}transport\_company\_ID & VARCHAR & primarni ključ tablice \\ 
				\hline
				name & VARCHAR & ime firme \\
				\hline 
				phone\_number & VARCHAR & felefonski broj firme \\
				\hline
				email & VARCHAR & email firme \\
				\hline 
			\end{longtblr}
			
			\noindent
			\textbf{TransportVehicle} Ovaj entitet sadržava informacije o nekom transportnom vozilu. Entitet sadrži atribute: transport\_comapny\_ID, name, phone\_number, email. Ovaj entitet je u \textit{Many-to-One} vezi s entitetom TransportCompany preko atributa transport\_company\_ID. I u \textit{One-to-Many} vezi s entitetom TransportBooking preko atributa transport\_vehicle\_ID.
			\begin{longtblr}[
				label=none,
				entry=none
				]{
					width = \textwidth,
					colspec={|X[11,l]|X[6, l]|X[20, l]|}, 
					rowhead = 1,
				} %definicija širine tablice, širine stupaca, poravnanje i broja redaka naslova tablice
				\hline 
				\SetCell[c=3]{c}{\textbf{TransportVehicle}}\\ 
				\hline[3pt]
				\SetCell{LightGreen}transport\_vehicle\_ID & VARCHAR & primarni ključ tablice \\ 
				\hline
				type & VARCHAR & vrsta vozila \\
				\hline 
				capacity & INT & kapacitet vozila \\
				\hline
				\SetCell{LightBlue} transport\_company\_ID	& VARCHAR & transportna firma kojoj pripada ovo vozilo \\
				\hline 
			\end{longtblr}
			
			\noindent
			\textbf{TransportBooking} Ovo je slabi vezni entitet koji označava prijevoz pacijenta. Ujedinjuje tri entiteta iz kojih saznajem kojeg pacijenta treba voziti, od kuda će biti transportiran, koje vozilo će se korisiti i do koje klinike će se transportirati. Entitet sadrži atribute: transport\_vehicle\_ID, accommodation\_booking\_ID, medical\_appointment\_ID. Entitet je u \textit{Many-to-One} vezi s entitetom TransportVehicle preko atributa transport\_vehicle\_ID. U \textit{Many-to-One} vezi s entitetom AccommodationBooking preko atributa accommodaton\_booking\_ID. I u \textit{Many-to-One} vezi s entitetom MedicalAppointment preko atributa medical\_appointment\_ID.
			\begin{longtblr}[
				label=none,
				entry=none
				]{
					width = \textwidth,
					colspec={|X[11,l]|X[6, l]|X[20, l]|}, 
					rowhead = 1,
				} %definicija širine tablice, širine stupaca, poravnanje i broja redaka naslova tablice
				\hline 
				\SetCell{LightBlue} transport\_vehicle\_ID & VARCHAR & vozilo koje obavlja ovaj projevoz \\
				\hline 
				\SetCell{LightBlue} accommodation\_
				booking\_ID & VARCHAR & poveznica na smještaj i pacijenta koje treba prevesti iz tog smještaja \\
				\hline
				\SetCell{LightBlue} medical\_
				appointment\_ID & VARCHAR & poveznica na medicinski tretman koji nam daje informaciju gdje pacijenta treba voziti \\
				\hline
			\end{longtblr}
			
			\noindent
			\textbf{MedicalAppointment} Ovaj entitet korisimo za spremanje podataka o medicinskim tretmanima nekog pacijenta. Informacije o ovom ćemo dobiti iz medicinskog sustava, ali ako nije moguće odmah napravit TransportBooking zbog npr. nedostatka vozila, želimo pospremiti informacije o tretmanu. Entitet sadrži atribute: medical\_appointment\_ID, PIN, clinic\_address, start\_datetime, end\_datetime. Entitet je u \textit{One-to-One} vezi s entitetom TransportBooking preko atributa medical\_appointment\_ID. 
			\begin{longtblr}[
				label=none,
				entry=none
				]{
					width = \textwidth,
					colspec={|X[10,l]|X[6, l]|X[20, l]|}, 
					rowhead = 1,
				} %definicija širine tablice, širine stupaca, poravnanje i broja redaka naslova tablice
				\hline 
				\SetCell[c=3]{c}{\textbf{MedicalAppointment}}\\ 
				\hline[3pt]
				\SetCell{LightGreen}medical\_
				appointment\_ID & VARCHAR & primarni ključ tablice \\ 
				\hline
				PIN & VARCHAR & Personal Identification Number pacijenta za kojeg je ovo medicinski tretman \\
				\hline 
				clinic\_address & VARCHAR & adresa klinike u kojoj je tretman \\
				\hline
				start\_datetime & DATETIME & datum i vrijeme početka tretmana \\
				\hline
				end\_datetime & DATETIME & datum i vrijeme kraja tretmana \\
				\hline
			\end{longtblr}
			
			\noindent
			\textbf{AdminRole} Ovaj entitet označava role admina. Sadrži atribute, admin\_role\_ID i name. Entitet je u \textit{One-to-Many} vezi s entitetom AdminRoles preko atributa admin\_role\_ID. 
			\begin{longtblr}[
				label=none,
				entry=none
				]{
					width = \textwidth,
					colspec={|X[6,l]|X[6, l]|X[20, l]|}, 
					rowhead = 1,
				} %definicija širine tablice, širine stupaca, poravnanje i broja redaka naslova tablice
				\hline 
				\SetCell[c=3]{c}{\textbf{AdminRole}}\\ 
				\hline[3pt]
				\SetCell{LightGreen}admin\_role\_ID & VARCHAR & primarni ključ tablice \\ 
				\hline
				name & VARCHAR & deskriptivni naziv role \\
				\hline 
			\end{longtblr}
		
			
			\noindent
			\textbf{Admin} Ovaj entitet sadrži podatke o adminima. Entitet sadrži atribute: admin\_ID, email, first\_name, last\_name. Entitet je u \textit{One-to-Many} vezi s entitetom AdminRoles preko atributa admin\_ID. 
			\begin{longtblr}[
				label=none,
				entry=none
				]{
					width = \textwidth,
					colspec={|X[6,l]|X[6, l]|X[20, l]|}, 
					rowhead = 1,
				} %definicija širine tablice, širine stupaca, poravnanje i broja redaka naslova tablice
				\hline 
				\SetCell[c=3]{c}{\textbf{Admin}}\\ 
				\hline[3pt]
				\SetCell{LightGreen}admin\_ID & VARCHAR & primarni ključ tablice \\ 
				\hline
				email & VARCHAR & email admina \\
				\hline 
				first\_name & VARCHAR & ime admina \\
				\hline
				last\_name & VARCHAR & prezime admina \\
				\hline 
			\end{longtblr}
			
			\noindent 
			\textbf{AdminRoles} Ovaj vezni entitet služi kao spoj nekog admina s njegovom rolom. Atributi entiteta su: admin\_role\_ID i admin\_ID. Entitet je u \textit{Many-to-One} vezi s entitetom AdminRole preko atributa admin\_role\_ID. I entitet je u \textit{Many-to-One} vezi s entitetom Admin preko atributa admin\_ID.
			\begin{longtblr}[
				label=none,
				entry=none
				]{
					width = \textwidth,
					colspec={|X[6,l]|X[6, l]|X[20, l]|}, 
					rowhead = 1,
				} %definicija širine tablice, širine stupaca, poravnanje i broja redaka naslova tablice
				\hline 
				\SetCell[c=3]{c}{\textbf{AdminRoles}}\\ 
				\hline[3pt]
				\SetCell{LightBlue}admin\_role\_ID	& VARCHAR & poveznica na rolu ovog admina \\
				\hline 
				\SetCell{LightBlue}admin\_ID & VARCHAR & poveznica na kojeg admina se odnosi ova rola \\
				\hline
			\end{longtblr}			
				
			
			\subsection{Dijagram baze podataka}

				\begin{figure}[H]
					\includegraphics[scale=0.32]{slike/dijagram_baze.png} %veličina slike u odnosu na originalnu datoteku i pozicija slike
					\centering
					\caption{ER Dijagram baze podataka}
					\label{fig:dijagram_baze_podataka}
				\end{figure}
			
			\eject
			
			
		\section{Dijagram razreda}
		
			\textit{Potrebno je priložiti dijagram razreda s pripadajućim opisom. Zbog preglednosti je moguće dijagram razlomiti na više njih, ali moraju biti grupirani prema sličnim razinama apstrakcije i srodnim funkcionalnostima.}\\
			
			\textbf{\textit{dio 1. revizije}}\\
			
			\textit{Prilikom prve predaje projekta, potrebno je priložiti potpuno razrađen dijagram razreda vezan uz \textbf{generičku funkcionalnost} sustava. Ostale funkcionalnosti trebaju biti idejno razrađene u dijagramu sa sljedećim komponentama: nazivi razreda, nazivi metoda i vrste pristupa metodama (npr. javni, zaštićeni), nazivi atributa razreda, veze i odnosi između razreda.}\\
			
			\textbf{\textit{dio 2. revizije}}\\			
			
			\textit{Prilikom druge predaje projekta dijagram razreda i opisi moraju odgovarati stvarnom stanju implementacije}
			
			
			
			\eject
		
		\section{Dijagram stanja}
			
			
			\textbf{\textit{dio 2. revizije}}\\
			
			\textit{Potrebno je priložiti dijagram stanja i opisati ga. Dovoljan je jedan dijagram stanja koji prikazuje \textbf{značajan dio funkcionalnosti} sustava. Na primjer, stanja korisničkog sučelja i tijek korištenja neke ključne funkcionalnosti jesu značajan dio sustava, a registracija i prijava nisu. }
			
			
			\eject 
		
		\section{Dijagram aktivnosti}
			
			\textbf{\textit{dio 2. revizije}}\\
			
			 \textit{Potrebno je priložiti dijagram aktivnosti s pripadajućim opisom. Dijagram aktivnosti treba prikazivati značajan dio sustava.}
			
			\eject
		\section{Dijagram komponenti}
		
			\textbf{\textit{dio 2. revizije}}\\
		
			 \textit{Potrebno je priložiti dijagram komponenti s pripadajućim opisom. Dijagram komponenti treba prikazivati strukturu cijele aplikacije.}