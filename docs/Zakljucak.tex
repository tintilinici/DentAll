\chapter{Zaključak i budući rad}
		
		 Zadatak naše grupe bio je razvoj web aplikacije pod nazivom \textit{ “DentAll”}. Ideja je bila razviti rješenje za učinkovitim upravljanjem smještajem i prijevozom pacijenata zdravstvenog turizma. Nakon 12 tjedana timskog rada, ostvarili smo zadani cilj i projekt je završen. Projekt je imao tri faze.
		 
		 Prva faza je uključivala okupljanje tima, razgovor o idejama za aplikaciju, izražavanje pojedinačnih interesa i želja za radom u prvom ciklusu predaje.
		 Naglasak je bio na radu dokumentiranja zahtjeva te razvoju backenda. Kvalitetno oblikovanje obrazaca i dijagrama ( uključujući obrasce uporabe, sekvencijske dijagrame, model baze podataka i dijagrame razreda ) značajno je olakšalo daljnji implementacijski razvoj aplikacije. Timovi su paralelno radili na dokumentaciji i backendu, dok su u području frontenda definirane ideje i implementirani osnovni dijelovi.
		 
		 Druga faza je počela puno intenzivnije te je naglasak više bio na implementaciji same aplikacije. Jedan tim radio je ne frontendu te drugi na backendu. Temeljito izrađena prva faza projekta pridonijela je uštedi vremena tijekom izrade aplikacije, izbjegavajući moguće pogreške i povećavajući učinkovitost timskog rada. U ovoj fazi uspješno smo dovršili implementaciju cijele aplikacije.
		 
		 U trećoj i konačnoj fazi našeg projekta fokus je bio na dokumentaciji i izradi raznolikih UML dijagrama, detaljno ispitivanje sustava, identifikaciju i ispravak grešaka te implementaciji preostalih funkcionalnosti. U ovoj fazi svi članovi su radili na dokumentaciji. Ova je faza trajala do završetka projekta, prije konačne predaje i kolokviranja drugog ciklusa.
		 
		 Aplikaciju je moguće proširiti na mnogo načina. Jedna od mogućnosti bi bila dodavanje interakcije s pacijentima to jest omogućiti ocjenjivanje te komentiranje usluga.
		 
		 Sudjelovanje u ovom projektu bilo je korisno iskustvo za sve članove tima. Iznimno smo zadovoljni postignutim rezultatima i timskim radom koji je pridonio ostvarenju tih rezultata.
		 
		 
		
		\eject 